\chapter{Introdução}
\label{cap:introducao}

A \textit{web} começou como uma simples rede de troca de documentos, mas tem se tornado a
plataforma de aplicativos mais onipresente de todos os tempos, acessível em uma vasta
gama de sistemas operacionais e tipos de dispositivos. Por acidente histórico,
\textit{JavaScript} é a única linguagem de programação suportada nativamente na
\textit{web}, com exceção de algumas tecnologias disponíveis apenas através de
\textit{plugins}, como: \textit{ActiveX}, \textit{Java} ou \textit{Flash}.
Devido à "onipresença" do \textit{JavaScript}, rápidas melhorias de desempenho em máquinas
virtuais modernas têm sido necessárias, talvez por necessidade absoluta, acabou se
tornando um alvo de compilação para outras linguagens. Através do \textit{Emscripten},
mesmo programas escritos em \textit{C} e \textit{C++} podem ser compilados em um
subconjunto estilizado de baixo nível de \textit{JavaScript}, chamado \textit{asm.js}.
No entanto, \textit{JavaScript} tem desempenho inconsistente e uma série de outras
armadilhas, especialmente como um alvo de compilação. \cite{wapaper}

\textit{WebAssembly} é uma alternativa proposta que tem como um de seus objetivos resolver
problemas presentes na utilização de \textit{JavaScript}, e que tem ganho bastante espaço
entre desenvolvedores recentemente. \textit{WebAssembly} é um novo tipo de código, que
pode ser executado em navegadores modernos, trata-se de uma linguagem de baixo nível, como
\textit{assembly}, com um formato binário compacto que executa com performance quase
nativa e que fornece um novo alvo de compilação para linguagens como
\textit{C}/\textit{C++}, para que possam ser executadas na \textit{web}. Também foi
projetado para executar em conjunto com o \textit{JavaScript}, permitindo que ambos
trabalhem juntos. \cite{mdn-wa}

\textit{WebAssembly} trás consigo grandes implicações para a plataforma \textit{web}, pois
fornece uma maneira de executar códigos escritos em vários outros idiomas além de
\textit{JavaScript} na \textit{web}, além de possibilitar uma velocidade de execução
próxima da execução de um código nativo. Possui padrão aberto desenvolvido por um grupo
comunitário da W3C que inclui representantes de todos os principais fornecedores de
navegadores até então.

Neste trabalho é apresentada uma análise comparativa quantitativa, entre algoritmos
escritos em \textit{JavaScript} e \textit{WebAssembly}, baseando-se em medições de seus
tempos de execução, através de experimentos, sendo estes descritos na seção
\ref{chap:development}, após um embasamento teórico necessário para a compreensão deste
trabalho descrito na seção \ref{cap:fundamentacao-teorica}, e finalizando com a seção
\ref{chap:conclusion} onde são feitas as considerações finais.

\section{Objetivos}
\label{sec:goals}

\subsection{Objetivo Geral}
\label{sec:general-goal}

A proposta deste trabalho é realizar uma análise comparativa entre algoritmos escritos em
\textit{WebAssembly} e \textit{JavaScript}, utilizando como base seus tempos de execução
para identificar em que casos, códigos escritos em \textit{WebAssembly} possuem desempenho
superior a códigos escritos em \textit{JavaScript}.

\subsection{Objetivos Específicos}
\label{sec:spec-goals}

\begin{alineas}
    \item Estudar o processamento realizado internamente em um motor \textit{JavaScript},
    e como ele processa algoritmos escritos em \textit{WebAssembly} e \textit{JavaScript}.
    \item Elaborar um modelo de obtenção de métricas de desempenho de execução de códigos
    escritos em \textit{JavaScript} e \textit{WebAssembly}.
    \item Desenvolver uma prova de conceito que sirva como base de comparação entre
    algoritmos escritos em \textit{WebAssembly} e \textit{JavaScript}, objetivando
    analisar a performance destes.
    \item Realizar uma análise comparativa de desempenho entre algoritmos escritos
    em \textit{WebAssembly} e \textit{JavaScript}, utilizando os resultados obtidos na
    prova de conceito.
\end{alineas}