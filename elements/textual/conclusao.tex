\chapter{Conclusão}
\label{chap:conclusion}

\section{Considerações finais}
\label{ssec:final-considerations}

Este trabalho se dispôs a fazer uma análise de uma tecnologia emergente no mercado,
chamada de \textit{WebAssembly}, que tem como principal objetivo permitir que aplicações
de alto desempenho sejam executados na \textit{web}, porém não faz quaisquer pressupostos
específicos da \textit{web} ou requer quaisquer recursos específicos da \textit{web},
sendo assim pode ser empregado em outros ambientes.

Com base no que foi apresentado neste trabalho, pode-se observar que \textit{WebAssembly}
se destaca em relação a performance devido alguns aspectos diretamente ligados a seus
objetivos de projeto. Um ponto a ser destacado em relação a \textit{WebAssembly}, é que o
mesmo possui uma representação intermediária binária extremamente compacta, o que faz com
que sua transmissão seja mais eficiente que um arquivo textual comum, como por exemplo um
arquivo \textit{JavaScript}. Além disso, cada arquivo binário representa um único módulo
e é dividido em seções, e cada seção é dividida em funções. Isso significa que a latência
de carregamento de um arquivo pode ser minimizada, ao iniciar o processo de compilação a
medida em que as funções desse arquivo estão sendo recebidas. Junto com essa abordagem,
é possível paralelizar o processo de compilação, distribuindo o processamento dessas
funções, assim como é feito pelos motores \textit{V8} e \textit{SpiderMonkey}.

Foi analisado neste trabalho, por meio de experimentos, que a execução de um código
escrito em \textit{WebAssembly} possui performance superior a um código escrito em
\textit{JavaScript} na maioria dos casos, através da execução dos algoritmos:
\textit{ShellSort}, \textit{QuickSort} e o algoritmo que busca o n-ésimo termo da
sequência \textit{Fibonacci}. O motivo dessa eficiência pode ser compreendida devido
diferenças cruciais entre sua execução e a execução de um algoritmo escrito em
\textit{JavaScript}. Uma diferença importante entre um algoritmo escrito em
\textit{WebAssembly} e \textit{JavaScript}, é que enquanto um algoritmo em
\textit{JavaScript} precisa ser analisado, para posteriormente ser gerada a Árvore de
Sintaxe Abstrata e em seguida a representação intermediária, para só então possa ser
convertido em código de máquina, um algoritmo em \textit{WebAssembly} já é uma
representação intermediária, e ainda mais eficiente que a representação gerada para um
código \textit{JavaScript} em alguns motores, precisando apenas ser convertido em código
de máquina.

\section{Trabalhos futuros}
\label{ssec:future-works}

Como continuação deste trabalho, deseja-se obter mais métricas de desempenho na execução
de cada algoritmo, de forma que se consiga realizar uma análise mais precisa sobre
diferenças de performance entre \textit{WebAssembly} e \textit{JavaScript}. Além disso,
estabelecer uma metodologia de análise de desempenho voltada a \textit{WebAssembly},
através de uma biblioteca criada com esse propósito, em que seja possível informar as
funções a serem testadas e as regras dos dados de entrada e essa biblioteca execute as
funções passando os dados corretos obedecendo as regras estabelecidas e gere todas as
métricas obtidas com base na execução de cada função isoladamente.


\glsaddall