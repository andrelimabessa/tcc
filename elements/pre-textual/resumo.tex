\textit{JavaScript} é a única linguagem de programação suportada nativamente pelos
navegadores, sendo bastante criticada, dentre outros motivos por ser extremamente
dinâmica, o que a fez se tornar alvo de compilação para várias outras linguagens. Um
projeto que tem se destacado principalmente pelo seu desempenho de execução, e que será
abordado neste trabalho, é o \textit{WebAssembly}, que também possui suporte nativo nos
navegadores modernos, e que possui a participação dos principais fornecedores de
navegadores até então. O propósito desta monografia é apresentar uma análise comparativa
quantitativa de desempenho entre \textit{WebAssembly} e \textit{JavaScript}, embasando-se
em seus tempos de execução para identificar em que casos, códigos escritos em
\textit{WebAssembly} possuem desempenho superior a códigos escritos em
\textit{JavaScript}.

% Separe as palavras-chave por ponto
\palavraschave{\textit{WebAssembly}. \textit{JavaScript}. Performance. Tempo de execução.}